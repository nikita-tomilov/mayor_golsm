Due to the recent growth in popularity of the Internet of Things solutions, the amount of data being captured, stored and transferred is also significantly increasing. The concept of edge devices allows buffering of the time-series measurement data to help mitigating the network issues. 
One of the options to safely buffer the data on such a device within the currently running application is to use some kind of embedded database. However, those can have poor performance, especially on embedded computers. That is why in this paper an alternative solution, which involves the LSM tree data structure, was advised.
The article describes the concept of an LSM tree-based storage for buffering time series data on an edge device within the Golang application. To demonstrate this concept, a GoLSM library was developed. Then, a comparative analysis to a traditional in-application data storage engine SQLite was performed. This research shows that the developed library provides faster data reads and data writes than SQLite as long as the timestamps in the time series data are linearly increasing, which is common for any data logging application.