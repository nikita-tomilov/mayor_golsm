Благодаря недавнему росту популярности Интернета Вещей, количество данных, которые собираются, хранятся и передаются, также существенно возросло. Концепция "краевых устройств"\\
позволяет буферизовать измерения, полученные с датчиков и представимые в виде временных рядов, чтобы избежать проблем в случае нестабильного соединения с Интернетом.
Одним из путей такой буферизации данных на краевом устройстве в рамках запущенного на нем приложения является использование встраиваемой базы данных. Однако, такие базы данных могут иметь слабую производительность, особенно на встраиваемых компьютерах. Это приводит к большим задержкам доступа к данным, что вредно для критически важных систем. Поэтому в этой статье предлагается альтернативное решение по буферизации временных рядов, использующее структуру данных LSM-дерево.
Данная статья описывает концепцию хранилища на основе LSM-дерева для буферизации временных рядов внутри приложения, написанного на языке программирования Go. Чтобы продемонстрировать эту концепцию, была разработана библиотека GoLSM. Также был произведен сравнительный анализ разработанного решения с традиционной встраиваемой базой данных SQLite. Исследование показало, что разработанная библиотека предоставляет более быстрый доступ на запись и чтение данных, чем SQLite, при условии, что отметки времени во временных рядов упорядочены по возрастанию, что типично для систем логирования данных.